\blu{10-2}

\section{$\Ga$ and $\si$}

%C^m function that agrees with the data. 
In the problem we're interested in, we're only given the values, not the Taylor polynomials at the points. It's a linear programming problem; in principle we can do it, but in time $N^3$. We'll get the time down to $N\log N$. We have to come up with a consistent set of Taylor polynomials. What might it be at one point?

%$\Ga$'s and $\si$'s. 
For $m,n>1$, we are given $E\sub \R^n$, $f:E\to \R$. We want to extend to a function $C^m(\R^n)$. 
We ask the right question: if we fix one particular point, what might the Taylor polynomial be at that one point?
Let 
\begin{align}
\Ga(x,M) &= \set{J_x}{F=f\text{ on }E, \ve{F}_{C^m}\le M}\sub \cal P.
\end{align}
This is a possibly empty convex set.
What is the approximate size and shape of $\Ga(x,M)$? 

It would be great if we can find the exact possible $C^m$ norm, but we don't know how; also there are lots of equivalent $C^m$ norms. Think of the inequality up to a constant. 
\begin{align}
\Ga(x,c_1M)&\sub
\Ga_{\text{computed}}(x,M) \sub \Ga (x,C_1M),\\
\Ga_{\text{computed}}(x,c_2M) \sub \Ga(x,M)\sub \Ga_{\text{computed}}(x,C_2M).
\end{align}
%given know $M$ approximately.

These sets have a lot of information. They're useful and even if we couldn't use them to compute interpolants we would still want to use them.
Suppose that I want to interpolate data. They come from some experiment or observations. The $f$ we're trying to find is presumably a smooth function of the data, but it's highly unlikely that the function is the interpolant with smallest norm. What do we believe? We believe it matches the data and the $C^m$ norm is not that large. 

Pick a point, what can we say? Suppose we observe where we are at discrete points in time. Pick some particular time, what can I say about position, velocity, acceleration? %I believe that the acceleration of the car is at most something. 
Given what you believe about your interpolant, we would like to know what kind of uncertainty there is in the interpolant. 

If we can decide whether $\Ga$ is empty, that tells us approximately the best possible norm of the interpolant.

When we prove theorems and construct interpolants, these $\Ga$'s are what we use to construct. 

First we look at $\Ga(x,M)$ at one particular point $x\in E$. We guess $\Ga_{\text{computed}}$, and prove correctness by constructing interpolants. We have to not find the jet at one point, but a family of jets that are mutually consistent.

%convex, possibly empty, contain true $\Ga$.
For $x\in E$ and $M$ fixed, we construct
\begin{align}
\Ga_\ell(x,M)\supset \Ga(x,M) 
\end{align}•
for $\ell>0$, convex, possibly empty.
Define by induction on $\ell$. Let different points in $E$ talk to each other and reduce the size to $\Ga_{\ell+1}(x,M)$. 

We induct on $\ell$. Let
\begin{align}
\Ga_0(x,M) :&= \set{P\in \cal P}{P(x)=f(x), |\pl^\al P(x)|\le M\text{ for }|\al|\le M}.
\end{align}
This is all the info that comes if you ignore all other points except $x$.

Suppose we know $\Ga_\ell(x,M)$ for all $x\in E$. Suppose $\Ga_\ell(x,M)\supset \Ga(x,M)$. We will define $\Ga_{\ell+1}(x,M)$ for all $x\in E$ such that $\Ga_\ell(x,M)\supset \Ga_{\ell+1}(x,M)\supset \Ga(x,M)$. 

Let $x\in E$, $P\in \Ga_\ell(x,M)$. Then let $P\in \Ga_{\ell+1}(x,M)$ iff for all $y\in E\bs \{x\}$ there exists $P'\in \Ga_\ell(y,M)$ such that 
\begin{align}
|\pl^\al(P-P')(x)| &\le M |x-y|^{m-|\al|}.
\end{align}
%Inequality would just be Taylor's Theorem. $m$ appears because the $C^m$ norm is at most $M$.
%possibly empty convex set.

How complicated are these sets? They are much too complicated, but we will go ahead and prove math theorems about them. We will not define the $\Ga_\ell$ this way, but differently to retain the key properties.

%insetad of $|\cdot|$ we can have a $\pm$. 
%these are linear constraints
The $\Ga_\ell$'s are convex polytopes defined by linear constraints. But $\Ga_\ell$ are polytopes defined by growing number of linear constraints. To get to the next steps, ontersect polytopes. The number of constraints defining them will grow very fast.
How to cope?
Compute them approximately. 

Define a blob, a 1-parameter family of growing convex sets.
%For each positive number $m$ I specify a convex set. 
We say two blobs are $C$-equivalent if they are the same up to a constant $C$. 

Before the $\Ga$'s are getting rapidly more complicated, but we can arrange things so they aren't.

The more serious point: every $y$ is talking to every $x$. It appears this definition requires $N^2$ steps. Fortunately, there is a clever way to use the well-separated pairs decomposition to compute something enough like them in time $N\log N$. 

\begin{thm}
$\Ga(x,M)\sub \Ga_\ell(x,M)$, and for $\ell_*=\ell_*(m,n)$, we have $\Ga_{\ell_*}(x,M) \sub \Ga(x,CM)$ for $C$ depending only on $m,n$.
\end{thm}
\footnote{So far $E$ has been finite. The analogue is not true for $C^m$ of infinite sets. One needs something more. That is the Glazer refinement. This is what Glazer refinement looks like for finite sets.}

The number of times you iterate to get something comparable to the true $\Ga$ is a fixed number.
%glazer refinement?
%version for infinite sets

%true for any family that satisfies...
Facts:
\begin{itemize}
\item
$\Ga_\ell(x,M)\sub \cal P$ is a possibly empty convex set.
\item %We know 
$\Ga_\ell(x,M)\sub \Ga_\ell(x,M')$ if $M\le M'$, by induction (conditions grow weaker as $M$ grows).
\item
Given $x,y\in E$, given $P\in \Ga_\ell(x,M)$, there exists
\begin{align}
P'&\in \Ga_{\ell-1}(y,M)&
\text{such that }|\pl^\al (P-P') (x)|&\le M(x,y)^{m-|\al|}.
\end{align}
We talk only about $\Ga$'s and their cousin the $\si$'s. %jet belongs to $\Ga_0$.
\end{itemize}•
%other properties
We want $F=C^m(\R^n)$ such that $\ve{F}_{C^m}\le CM$ and $J_x(F) \in \Ga_0(x,M)$ for all $x\in E$. 
%whole problem phrased in terms of $\Ga$'s.

There are two ways to construct $\Ga_\ell$'s. We show another. Then we introduce the related sets $\sigma$.

We use the finiteness and refined finiteness theorems.
\begin{thm}[Finiteness Theorem]\label{thm:fin}
Let $E\sub \R^n$, $\#E=N<\iy$, $f:E\to \R$. Given $M>0$, suppose that for every subset $S\sub E$ with at most $k^\#$ points (depending only on $m$, $n$), there exists $F^S\in C^m(\R^n)$ of norm $\le M$ such that $F^S=f$ on $S$. Then there exists $F\in C^m(\R^n)$ of norm $\le C M$ ($C$ depending only on $m$, $n$) such that $F=f$ on $E$.
%pure math world don't care about value of functoins, this is way in principle computes best possible norm by constant factor
%complexity increaseses as $s$ doesn, but we've capped
%can do by linalg
\end{thm}
%QF defined on linear subspace
%up to factor depends on $|S|$, controlled.
%For $F^S=f$ on $S$, $\ve{F^S}_{C^m}\le M$. 
%fixed size thanks to whitney
%...
%Suppose we were to use tis to make a calculation. 
%There's something more fundamental that is wrong: 
To use this we would need to look at $\sim N^{k^\#}$ sets.  %Let's say $k^\#$ is 100. Then we need to look at $\sim N^
We need a refinement. 

\begin{thm}[Refined finiteness theorem]\label{thm:rft}
Fix $m,n$. Let $E\sub \R^n$, $\#E = N$. Then there exist $S_1,\ldots ,S_L\sub E$ with the following properties
\begin{itemize}
\item
$\#(S_\ell)\le k^\#$ for each $\ell$,
\item
$L\le CN$, 
\item
Let $f:E\to \R$ and let $M>0$. Suppose that for each $\ell=1,\ldots, L$ there exists $F_\ell\in C^m(\R^n)$ with norm $\le M$ such that $F_\ell=f$ on $S_\ell$. Then there exists $F\in C^m(\R^n)$ of norm $\le CM$ such that $F=f$ on $E$.
 %instead of  ..., minimize quadratic norm...
\end{itemize}
where $k^\#$ and $C$ depend only on $m,n$.

Then $\max_{\ell=1,\ldots L}\ve{f}_{S_{\ell'}}\le \ve{f}_E\le C \max_{\ell=1,\ldots, L} \ve{f}_{S_\ell}$. The $S_1,\ldots, S_L$ can be computed from $E$ in $O(N\log N)$ computer operations.
\end{thm}

%SoC lecture
Digression on outliers: 
Suppose we collect data from a physics experiment, and the machine malfunctions or the technician falls asleep. We get data points that are completely wrong and should be discarded.
%undergrad physics
%had to perform experiments
%some of these things are due to me.
You discover that the smallest possible norm is enormous, and would like to discard some data. If you are allowed to discard a few to bring the interpolant way down, which should you ignore? As a consequence of the refined finiteness theorem, there is a theorem with an algorithm attached.
\begin{thm}
Fix $m,n$. Given $f:E\to \R$, $\#E=N$, there exists an enumeration of $E$, $E=\{x_1,\ldots, x_N\}$ such that the following holds.

Let $S\sub E$ and suppose there exists an interpolant of norm $\le M$ for $f|_{E\bs S}$, with $\#S=Z$.
Then $f|_{E\bs \{x_1,\ldots, x_{CZ}\}}$ has an interpolant with norm $\le CM$. 

The points $x_1,\ldots, x_N$ can be computed using $\le CN\log N$ operations per point.
\end{thm}
Suppose we have a contest against God to remove outliers. God instantaneously know the very best set of $Z$ outliers to remove. We are at a disadvantage. Let's cheat to give us a chance to prevail anyway. We can throw away 50 times as many points, and declare victory if our norm is within 50 times. Then we can win anyway.

It's remarkable this can be done at all; I don't think this is optimal. 

%I owe you: if you believe RFT then you can deal with outliers this way.
%for a long time, we won't prove RFT. FT is equivalent to $\Ga_\ell$ (or at least follows from). 
%reduce to discussion of $\Ga_\ell$. 
%what $\si$'s are
%complete list of basic properties of $\Ga$. 
%and some others. 
%ready to start proving the main theorem about the $\Ga$'s. in simplest nontrivial case, $C^2$ of the plane.