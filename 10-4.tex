\blu{10-4}

\subsection{$\si_\ell$'s}

We will cover
\begin{itemize}
\item
$\si$'s
\item
connection of finiteness theorem to $\Ga$'s
\item
Infinite $E$'s (new ingredients)
\item
The key properties of $\Ga$'s and $\si$'s.
\item 
Outliers
\end{itemize}

Let $f:E\to \R$, $E\sub \R^n$ finite, $m\ge 1$ fixed. Let $\Ga(x,M) = \set{J_x(F)}{F\in C^m \text{ with norm }\le M, F=f\text{ on }E}$. For $x\in E$, we defined $\Ga_\ell (x,M)$ by 
\begin{align}
\Ga_0(x,M) :&= \set{P\in \cal P}{P(x)=f(x), |\pl^\al P(x)|\le M\text{ for }|\al|\le M}\supset \Ga(x,M)\\
\Ga_{\ell+1}(x,M) :&= 
\set{P\in \Ga_\ell(x,M)}{\forall y\in E\bs \{x\}, \exists P'\in \Ga_\ell(y,M), |\pl^\al(P-P')(x)|\le M|x-y|^{m-|\al|}, |\al|\le m}.
\end{align}
If I have 2 interpolants for the same function, then their difference is an interpolant for 0. So the interpolants for 0 tell us how much arbitrariness there is in the interpolants. 
%dependence on M is trivial... 
%different M, trivial. dilate by M.

WLOG consider $M=1$.
Let 
\begin{align}
\si(x) &= \set{J_x(F)}{F\in \cal C^m, \text{ norm }\le 1, F=0\text{ on }E}\\
\si_0(x) &=\set{P}{\forall |\al|\le m,|\pl^\al P(x)|\le 1, P(x)=0}\\
\si_{\ell+1}(x)&=\set{P\in \si_\ell(x)}{\forall y\in E\bs \{x\}, \exists P'\in \si_\ell(y)\text{ such that }\forall |\al|\le m, |\pl^\al(P-P')(x)|\le |x-y|^{m-|\al|}}
\end{align}
For $P,P'\in \Ga(x,M)$, $P-P'\in 2M \cdot \si(x)$. For $P\in \Ga(x,M)$, $Q\in 2M\si(x)$, $P+Q\in \Ga(x,3M)$. 
If $\Ga(x,M_0)\ni P_0$ is nonempty, then for all $M\ge 5M_0$, 
\begin{align}
P_0+M\si(x) \sub \Ga(x,M) \sub P_0+10M\si(x)
\end{align}
To understand the $\Ga$'s,  we need to find the magnitude of the smallest $M_0$ such that $\Ga(x,M_0)$ is nonempty, and produce one element. 

For $P,Q\in \cal P$, define the pointwise product $P\odot_x Q= J_x(PQ)$, and $J_x(FG) = J_x(F)\odot_x J_x(G)$. For $P(x) = \sum_{|\al|\le m} a_\al x^\al$ and $Q(x) = \sum_{|\al|\le m} a_\be x^\be$, $PQ(x) = \sum_{|\al|,|\be|\le m} a_\al b_\be x^{\al+\be}$, $P\odot_0 Q = \sum_{|\al|+|\be|\le m} a_\al b_\be x^{\al+\be}$. 

For $\ve{F}_{C^m}\le 1$, $F=0$ on $E$, $J_x(F)=P$, %P times the harmless poly belongs to $\si$. We can deal with some harmful polys
apply this with the Whitney extension theorem. Interpolants work locally on different scales; patch them together. The cutoff functions will behave quite badly. 
%derives \le 
Consider $x=0$. 
Let $P=J_0(F)$. Suppose %$|\pl^\al F(0)|\le \de^{m-|\al|}$, 
$|\pl^\al F(0)|\le \de^{m-|\al|}$, $|\pl^\al Q(0)|\le \de^{-|\al|}$
%, $|\pl^\al F(0)|\le \de^{m-|\al|}$ 
for $\de\le 1$. If $P\in \si(0)$ then $Q\odot_0 P \in C\si(0)$, for $C$ depending only on $m$, $n$. 

There exists $\te\in C_0^\iy(\R^n)$ such that $\Supp\te\in B(0,\de)$, $J_0(\te)=Q$, $|\pl^\al \te|\le C\de^{-|\al|}$ everywhere, for $|\al|\le m$. 

What does this mean when $\de=1$? All the coefficients are bounded. There exists some $C^\iy$ function supported on the unit ball whose Taylor polynomial at 0 is $Q$ and whose derivatives are bounded. This is clear by multiplying by a cutoff function. For $\de\ne 1$, this follows from $\de=1$ by rescaling. 

Look at $F\cdot \te$. $\pl^\al(F\cdot \te)$ is a sum of terms
\begin{align}
|\pl^\be F(x) \pl^\ga \te(x) |
&\le C\de^{-|\ga|}
\end{align}
where $\be+\ga=\al$. %expand F as taylor poly around 0 plus rem
If $|\pl^\be F(0)|\le \de^{m-|\be|}$ for all $|\be|\le m$ and $|\pl^{\wh\be} F(x)|\le 1$  for $|\wh \be |=m$, then $|\pl^\be F(x)|\le C\de^{m-|\be|}$ for $|\be|\le m$. 

%FIX: above is somewhat confusing

(To  check effortlessly something is scale-invariant, assign units.)
%dim correct, so independent of scaling.
%derivs of up to order m
%F\cdot \te 
%C^m norm dominated by constant

We've proven that if $P\in \si(0)$, then $Q\odot_0P \in C\si(0)$.


\begin{df}\label{d:wc}
Let $\si\sub \cal P$ be any convex symmetric set. Let $x\in \R^n$, $C_w>0$, $\de_{\max}>0$. 

Then $\si$ is \vocab{Whitney convex} at $x$, with Whitney constant $C_w$ below length scale $\de_{\max}$, iff for all $P,Q\in \cal P$, $0<\de\le \de_{\max}$, if $|\pl^\al P(x)|\le \de^{m-|\al|}$ and $|\pl^\al Q(x)| \le \de^{-|\al|}$ for all $|\al|\le m$, and if also $P\in \si$, then $P\odot_x Q \in C_w\si$. 
\end{df}
If I don't specify the length scale, it is 1.
\begin{rem}
%$\si(x)$ is Whitney-convex at $x$ with Whitney constant 
Given $E\sub \R^n$, construct $\si(x)$ from interpolants. Then for all $x\in \R^n$, $\si(x)$ is Whitney convex at $x$, with Whitney constant defined by $m,n$ below length scale $\de_{\max}=1$. 
%diff dim and taking max - unnatural. reflected in: there is length scale.
%C^m(\dot) sup at order precisely $m$>
\end{rem}
% FIX what is this remark saying?
The definition of the $C^m$ norm is somewhat unnatural---taking different dimensional quantities (different order derivatives) and taking the max. This unnaturalness is reflected in the length scale. We can also define $\dot C^m$ norm which takes a sup over derivatives with order precisely $m$.%, and prove an analogous 

Check by induction on $\ell$ that $\si_{\ell}(x)$ is also Whitney convex with constant that depends on $\ell,m,n$ below length scale 1. 

Let $x\in E$, $M>0$, $\ell\ge 0$. Then $\Ga_\ell(x,M)$ is a (possibly empty) convex subset of $\cal P$, $\si_\ell(x)$ is a convex symmetric set in $\cal P$. For $M\ll M'$, %with a small constant determined by $m,n,\ell$, 
$\Ga_\ell(x,M)\sub \Ga_\ell(x,M')$, $\Ga_{\ell+1}(x,M) \sub \Ga_\ell(x,M)$. 

If  $x,y\in E$, $P \in \Ga_\ell(x,M)$, $\ell\ge 1$, then there exists $P'\in \Ga_{\ell-1}(y,CM)$, $C$, depending only on $m,n,\ell$ , such that $|\pl^\al(P-P')(x)|\le CM|x-y|^{m-|\al|}$, for $|\al|\le m$. 

If $P,P'\in \Ga_\ell(x,M)$, then $P-P'\in CM\si_\ell(x)$. If $P\in \Ga_\ell(x,M)$, $Q\in M\si_\ell(x)$ then $P+Q\in \Ga_\ell(x,CM)$.
$\si_\ell(x)$ is Whitney convex at $x$, with Whitney constant $C$ below length scale 1.
%et come from these properties

%finitenes stheorem and refined finiteness theorem

We talk about the finiteness theorem. %If you understand the $\ga$'s. If the theorem on the $\ga$'s is 
%if understand $\Ga_\ell$'s well enough, FT follows. 
%unif bound on $C^m$ norm

\subsection{Proof of finiteness theorem given non-emptiness}

Define 
\begin{align}
\Ga(x,M,S) &= \set{J_x(F)}{F=f\text{ on }S, \ve{F}_{C^m}\le M}\\
\wh \Ga_\ell(x,M) &= \bigcap_{S\sub E, \#S\le k_\ell} \Ga(x,M,S)\sub \cal P
\end{align}
convex. We need to prove this is nonempty. We use the following basic theorem in convex geometry.
\begin{thm}[Helly's Theorem]\label{thm:helly}
Suppose that $K_1,\ldots, K_N$ are convex (and not necessarily compact) subsets of $\R^D$. 
Suppose that any $D+1$ of the $K_i$ have a point in common. Then $K_1\cap \cdots \cap K_N\ne \phi$.
%3 minutes or so. That resonates with theorem up there. 
\end{thm}
If there are infinitely many convex compact sets, then this is still true.
If you take infinitely many convex sets, not necessarily compact, then this is not true; for a counterexample take $(0,\rc n]$. 
\begin{proof}
Given $L\ge D+1$, we show that if any $L$ of the $K$'s intersect, then also $L+1$ of the $K$'s intersect.

Consider $K_1,\ldots, K_{L+1}$. For each $i=1,\ldots, L+1$, pick $x_i$ in the intersection of all these $K$'s except $K_i$. 
%proof on wikipedia
We obtain $x_1,\ldots, x_{L+1}$. 

Look for coefficients $\be_1,\ldots, \be_{L+1}\in \R$ such that 
\begin{align}
\be_1+\cdots + \be_{L+1}&=0\\
\be_1x_1+\cdots + \be_{L+1}x_{L+1}&=0.
\end{align}
There are $D+1$ equations, and at least $D+2$ unknowns, so there is a nonzero solution. Put the positive ones on the LHS and the negative ones on the RHS: after possibly relabeling,
\begin{align}
\la_1x_1+\cdots +\la_ax_a &= \mu_1x_{a+1}+\cdots + \mu_b x_{a+b}\\
\la_1+\cdots + \la_a &= \mu_1+\cdots +\mu_b.
\end{align}
The $\la$'s and $\mu$'s are nonnegative and not all 0; we can rescale so that 
\begin{align}
\la_1+\cdots + \la_a &= \mu_1+\cdots +\mu_b=1.
\end{align}
We claim in $\la_1x_1+\cdots +\la_ax_a = \mu_1x_{a+1}+\cdots + \mu_b x_{a+b} \in \bigcap_{i=1}^{L+1}K_i$. The LHS lies in 
$\bigcap_{i=a+1}^{a+b} K_i$ and the RHS lies in $\bigcap_{i=1}^a K_i$. This completes the induction step and proves the theorem. 
\end{proof}






