
\blu{11-13}

Given $x_0\in E\cap 5(Q^0)^+$, $p_0\in \Ga_\ell(x_0,M_0)$, the problem is to produce $F\in C^m \pa{\fc{65}{64} Q^0}$ such that $|\pl^\al F|\le CM_0$ on $\fc{65}{64}Q^0$ for $|\al|=m$. We want
\begin{itemize}
\item
\begin{align}
|\pl^\al(F-P_0)|&\le C\de_{Q^0}^{m-|\al|}
\end{align}
on $\fc{65}{64}Q^0$ for $|\al|\le m-1$. 
(It doesn't need to agree exactly.)
\item
$J_xF\in \Ga_0(x,CM_0)$ for all $x\in E\cap \fc{65}{64}Q^0$. 
\end{itemize}
This is LIP$(Q^0,x_0,P_0)$. We have to prove every LIP can be solved. Once this is true, there is an interpolant provided that the suitable $\Ga_\ell$ is not empty. The goal is to prove such problems always have a solution. The proof will be constructive, and the constructions will be linear. %depend linearly on the data.
Prove by inductions on labels, which have to do with size and shape of $\si$'s. 

LIP$(Q^0,x_0,P_0)$ carries a label $\cal A$ iff for all $x\in 5Q^0$, $\si_\ell(x)$ has an $(\cal A, \de_{Q_0}, C)$-basis. %$\ell=?$
All $\cal A$ we consider will be monotonic: if $\al\in \cal A$, $\ga\in \cal M$, and $\al+\ga\in \cal M$, then $\al+\ga\in \cal A$. 
Let $\si$ be a convex symmetric set in $\cal P$. An $(\cal A, \de, C)$-basis for $\si$ at $x$ is a collection of polynomials $(P_\al)_{\al\in \cal A}$ such that 
\begin{itemize}
\item
$\pl^\be P_\al(x)=\de_{\be \al}$ if $\al,\be\in \cal A$
\item
$|\pl^\be P_\al(x)|\le C_1  \de_{Q^0}^{|\al|-|\be|}$ for $\al\in \cal A$, $\be \in \cal M$.
\item
$\de^{m-|\al|}P_\al \in C_1\si$ for $\al\in \cal A$. 
\end{itemize}
We show inductively that if LIP carries a label $\cal A$, then we can solve it. 
Every LIP carries the label of the empty set, as the conditions are vacuously true. 

We patch things together with partitions of unity, so will have to multiply things by smooth functions. Things that have an $\cal A$-basis behave well with multiplication. %PoU

Let $\ell(\cal A) = 1+3\#\set{\cal A'}{\cal A'<\cal A}$. I will take $\ell=\ell(\cal A)$; this is how I attach labels. (For all $x\in 5Q^0$, $\si_{\ell(\cal A)}$ has an $(\cal A, \de_{Q_0},C)$-basis.) There is extra information that helps us construct interpolants. We will take the general case and reduce it to a case with extra information, until we reduce to the trivial case. When $\cal A=\cal M$, $\ell(\cal A)=1$. Take $C(\cal A)$ large enough for the base case $\cal A=\cal M$. %so that $\si_1(x)$
%F=P_0$.
%have to do something 3 times

We make a Calderon-Zygmund decomposition of $\fc{65}{64}Q_0$. I need to tell you what the criteria is for being happy (when cutting the cube).
Suppose $\cal A$ is given, $\cal A\ne \cal M$. Assume that LIP can be solved for $\cal A$. 
%We are happy if LIP for the smaller cube $Q$ can be solved with a smaller label, or $E\cap 5Q=\phi$. %, then we are happy
%So: 
We are happy, and stop cutting $Q$ if either 
\begin{itemize}
\item
$E\cap 5Q$ contains at most one point.
\item
LIP$(Q,x,P)$ carries a label $\cal A'<\cal A$. ($P$ is unspecified for now. The definition of carrying a label has nothing to do with $P$.)
%doesn't depend on choice of $P_0$. 
\end{itemize}•
Is this a nice CZ decomposition? Do we ever stop cutting. There's always the danger of never being happy. $E$ is finite set, so for $Q$ sufficiently small, $E\cap 5Q$ contains at most one point, and we will stop cutting. %where we use finite
A good CZ rule should lead to good geometry. This rule does not lead to good geometry. %I'm pretty sure it doesn't.
This is fatal, so we will modify it slightly to save the geometry. We define a weak $(\cal A,\de,C_1)$ basis.
\begin{df}%\be< \al say nothing about derivs
An \vocab{weak $(\cal A, \de, C)$-basis} for $\si$ at $x$ is a collection of polynomials $(P_\al)_{\al\in \cal A}$ such that 
\begin{itemize}
\item
$\pl^\be P_\al(x)=\de_{\be \al}$ if $\al,\be\in \cal A$
\item
$|\pl^\be P_\al(x)|\le C_1 \de_{Q^0}^{|\al|-|\be|}$ for $\al\in \cal A$, $\be \in \cal M$ if $\be \ge \al$.
\item
$\de^{m-|\al|}P_\al \in C_1\si$ for $\al\in \cal A$. 
\end{itemize}
LIP$(Q^0,x_0,P_0)$ weakly carries label $\cal A$ if for all $x\in 5Q^0$, $\si_{\ell(\cal A)}(x)$ has a weak $\cal A,\de,C)$-basis. 
\end{df}
Suppose we have 2 CZ cubes that touch each other but one is much smaller. Then the parent of the smaller cube is unhappy; we produce a contradiction.
If we have 2 cubes, ahd one is half the size of the other, then 5 times the smaller one is contained in 5 times the bigger one.
%if happy for trivial reasons, then 
%Every point of 
So $5E\cap \pat{small}\subeq 5 E \cap \pat{big}$. In the 2nd and 3rd conditions for $\de$ the sidelength of the bigger fellow.
The $\de$ of the smaller cube is smaller,  in 3,this helps us. %If 2 and 3 are true, 
%makes easier.
If 2, it depends on $|\al|-|\be|$. For $\de$ smaller, we ask for a larger upper bound, a weaker. $\be - \al\le 0$ . %is K. 
%For weak bases
%$E\cap 5Q^+$. If emptyll
%pic o

Given $x_0$, $P_0\i \Ga_{\ell(\cal A)}(x_0,M_0)$, I can find $P\in \Ga_{\ell(\cal A)-1}(x_0,M_0)$ such that $|\pl^\al (P-P_0)(x_0)| \le CM_0|x-x_0|^{m-|\al|}$. 
We only have a weak basis, but we will see it doesn't matter. Pretending this doesn't matter...

%On a slightly dilated version of each $Q$, 
There is 2 glaring faults in the plan: what we have is a weak basis, but we need a basis. We can solve each LIP, but when we patch on a Whitney PoU, they have to agree very well, but there's no reason why they should. We have not been careful enough to pick the $P$. Why $\si$'s matter: in directions where $\si$ is big we need to worry about consistency, in directions where $\si$ is small we need to worry about solving the LIP.  %room to maneuver to pick wrong guy. Let us ... 
%constraints satisfed, 
Pick $P$ such that $|\pl^\al (P-P_0)|\equiv 0$ for all $\al\in \cal A$. This is not immediately obvious, but one can do that.
%by norm factor, belong to si
Make room to manuever inside $\Ga$ by adding suitable multiples.
%Once we do that, if two CZ cubes td
%PoU
%If label not monotonic, find smaller that is monotonic, carrying $\al$ is close to satisfied.

%local LIP has to look at neighbor to decide wither 

%look at on 

%ind basis strong basis

Why is weak/strong basis not a failure?
Suppose I give you a weak $\cal A$-basis.
%\cal A'$ basis. 
%\cal A'
%'' basis 
%induct on weak basis)

%If you have a convex set and it has a weak $\cal A$-basis, then it has a $\cal B$-basis. What happens if we have...

Let $\cal A=\{\al\}$, $\si$ be $(\cal A,L,C_1)$-basis at $\cal A$. 
\begin{itemize}
\item
$\pl^\al P(0)=1$.                                                                                                                                                                                                                                                        
\item
$|\pl^\be P(0)|\le C_1$ for $\be \ge \al$.
\item
$P \in  C_1 \si$.
\end{itemize}
Suppose $|\pl^\be P(0)|$ is maximal for $\be=\be_0<\al$. In particular, $|\pl^{\be_0}P(0)|>1$. 

Claim: $\wt P= \fc{P}{(\pl^{\be_0}P)(0)}$ is an $(\wt A=\{\be_0\}, 1, C_1)$-basis for $\si$ at $x_0$. %by maximality?
\begin{itemize}
\item
$\pl^{\be_0} \wt P(0)=1$.
\item
$|\pl^{\be_1}\wt P(0)| = \fc{|\pl^\be P(0)|}{\pl^{\be_0}P(0)} \le 1\le C_1$ for all $\be$. 
\item
$\wt P\in C_1\si$. 
\end{itemize}
We've rescaled and relabeled it. 
I claim that $\wt{\cal A}< \cal A$. This follows from $\be_0<\al$.

From a weak $\cal A$-basis, we get a $\wt{\cal A}$-basis for label that precedes $\cal A$.

That's the germ of the idea, for singletons. We need to do this for all $\al$'s simultaneously. %next time
%relabeling lemma: for any label, if I give weak basis, then there exists strong basis, for label that is $\le$.

