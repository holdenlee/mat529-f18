\blu{9-25}

\subsection{Well-separated pairs decomposition, proof}

The idea is simple. Suppose that $E\subeq Q_\nu\sub \R^n$. Let's look at $
Q^\circ \times Q^\circ$. 
Make a Whitney decomposition of the complement of the diagonal. Each is comparable to its distance from the diagonal. We make these cubes $Q_\nu=Q_\nu'\times Q_\nu''$ such that 
%\R^n\times \R^n$. Let's say $
\begin{align}
d(Q_\nu, \text{Diag}) \sim 10^5 \de_{Q_\nu}. 
\end{align}
I take $E_\nu'\times E_\nu'' = E\times E \cap Q_\nu'\times Q_\nu''$, by setting
\begin{align}
E_\nu' &= E\cap Q_\nu'\\
E_\nu'' &= E\cap Q_\nu''
\end{align}
Then
%these guys well-sep
%s.t. nontrivial
\begin{align}
d(E_\nu',E_\nu'') &\ge d(Q_\nu', Q_\nu'')\\
&\sim 10^5 (\diam Q_\nu' + \diam Q_\nu'')\\
&\ge 10^5 (\diam E_\nu' + \diam E_\nu'')
\end{align}•
There's a slight embarrassment that we need $\nu_{\max}\le CN$, whereas the number of Whitney cubes are infinitely. Fortunately, most of these $E_\nu',E_\nu''$ are empty.
%one point, cut infinitely

\begin{lem}
The number of nonempty $E_\nu'\times E_\nu''$ is at most $CN$. 
\end{lem}
%10^5 times diameter of ...
%dyadic cube 10^5 times larger that contains both.
%may or may not be a dyadic cube.

%Let's count a Whitney cube if there is a cube that contains both of them and with side length at most a large constant times their size.
\begin{proof}
Suppose $Q$ is a dyadic cube with $Q_\nu'$ and $Q_\nu''$ are contained in $Q$ ($Q_\nu',Q_\nu''\sub Q$) and
\begin{align}
\de_Q &< 2^{20} \de_{Q_\nu'} + 2^{20}\de_{Q_\nu''}\\
E\cap Q_\nu', E \cap Q_\nu'' &\ne \phi.
\end{align}•
Then we say that $Q$ accounts for $Q_\nu'$ and $Q_\nu''$. 

Unfortunately, it could be that no cube accounts for the pair.
Consider 2 tiny intervals, one slightly $<\rc2$ and the other slightly $>\rc 2$, so that the distance between them is $10^6$ times the side length. The smallest dyadic interval that contains them is the whole unit interval

This seems an unusual situation; in a typical case it will not happen.

We first count the pairs that are accounted by something, and then fix it up. 
%I'd like to prove
\begin{align}
\sum \# \set{Q_\nu'\times Q_\nu''}{\text{some dyadic $Q$ accounts for $Q_\nu',Q_\nu''$}}
&\le CN.
\end{align}
I'll look at all dyadic $Q$ that contain points of $E$. Under inclusion, such cubes form a tree $T$; stop cutting when $Q$ contains a single point of $E$. %The leaves are those that contain a single point of $E$ (we stop cutting then). 

If there is a cube $Q$ that accounts for $Q_\nu',Q_\nu''$, then after a bounded number of cuts, there is a branch. (By definition, $Q$ is at most a constant times larger than $Q_\nu',Q_\nu''$.)
%then there is some subcube, after 
The number of dyadic cubes that account for something is $\le $ a constant times the number of nodes in the graph where the graph branches. 

We need to estimate the number of branch points in the tree. Elementary fact about tree: the number of branch points is the number of leaves minus 1. To see this, induct on the size of the graph.

The number of leaves is the number of points of $E$, so the number of branch points is $\le \# E-1$.

Consider pairs $Q$ that account for something, and branch points $\le 11$ levels below it. For every $Q$ that accounts for something, there is a branch point $\le 11$ levels below it. For each branch point there are at most 11 dyadic $Q$ above it. 
 Hence the number of dyadic $Q$ that account for some $Q_\nu'\times Q_\nu''$ is less than $11\# E$.

%each $Q$ accounts for a bounded number.
%the number of $Q$ that acct for anything at most E.

%at most 11 levels down there is a branch.
%$Q$ that account for something, and branch points $\le 11$ levels. For every $Q$ that accounts for something, there is a branch point $\le 11$ . For each branch point there are at most 11 dyadic $Q$ above it. 

It can happen there are pairs not accounted for. %Imagine lots of dyadic grids, $[-2^k,2^k]\times \cdots \time [-2^k,2^k]$.
Let $\cal D_0$ be the set of all dyadic cubes. For $\xi\in \R^n$, let $D_\xi$ be all $Q+\xi$, $Q\in \cal D_0$. 
For a cube not accounted for in $\cal D_0$, consider it in $\cal D_\xi$. 
We can talk about whether $Q$ accounts for something with respect to $D_\xi$. 
For any fixed $\xi$, the number of $Q'\times Q''$ accounted for by $D_\xi$ is $\le C \#E$. 
%11
%common side length, 

Picking $\xi$ at random, what is the probability that two fixed cubes $Q_\nu', Q_\nu''$ lie on different squares of the $2^{11}$-times-larger grid and hence aren't accounted for? Unlikely. %Very unlikely. 
We have $Q_\nu',Q_\nu''$ is accounted for by some $Q=\cal D_\xi$ with probability $>\rc 2$. 

We estimate the number of 
pairs $(Q_\nu'\times Q_\nu'', \xi)$ such that $Q$ accounts for $Q_\nu'\times Q_\nu''$ in $\cal D_\xi$ in 2 ways.  
%in 2 different ways:
\begin{align}
\E_\xi (\cdot) &=  \sum_{Q_\nu'\times Q_\nu''} \Pj(Q_\nu'\times Q_\nu'' \text{ is accounted for by some }Q\in \cal D_\xi) \ge \rc 2\#\{Q_\nu'\times Q_\nu''\}.
\\
\E_\xi(\cdot) &\le C\# E.
\end{align}
%this one has a special structure, because 
\end{proof}
Note in each of the Cartesian products $E_\nu' \times E_\nu''$, each of $E_\nu' ,E_\nu''$ is the intersection of $E$ with a cube. 

One convenient way to write down the well-separated pairs decomposition theorem is to write down the cube.

How to compute it efficiently? It can be; the number of steps is $O(N\log N)$. We write down all the relevant cubes; for each $Q_\nu'\times Q_\nu''$, we exhibit one particular point.

%lower bounds?

%large finite set?

Let $E\sub \R^n$. For all $x\in E$ given $P^x\in \cal P$, does there exist $F\in C^m$ such that $J_x F = P^x$ for all $x\in E$? If so, how small can we take the $C^m$ norm of $F$?

%can be implemented efficiently

First do one-time work, $O(N\log N)$. Then we can answer queries.

%linear programming problem

Let $E\sub \R^n$, $f:E\to \R$, $\# E = N$. We want $F\in C^m(\R^n)$ such that $F=f$ on $E$, with norm of $F''$ as small as possible.

We look for $[(P^x)_{x\in E},M]$, satisfying the following constraints with $M$ as small as possible.
\begin{align}
(P^x)(x) &= f(x) &\forall x\in E\\
|\pl^\al (P^x)(x) | &\le M &\forall x\in E, |\al|\le m\\
|\pl^\al (P^x - P^y)(x)| &\le M|x-y|^{m-|\al|} &\forall x,y\in E\text{ distinct, }|\al|\le m.
\end{align}•
For finite $E$, the $o(\cdot)$ condition is vacuous.

This problem can be reduced to linear programming. This is a big LP: there are $N$ constraints, $N$ constraints, and $N^2$ constraints in the three sets. The well-separated pairs decomposition reduces this to $O(N)$ constraints.

In the WSPD, we get $E_\nu' \times E_\nu''$, $\nu=1,\ldots, \nu_{\max}$, $\nu_{\max}\le CN$. Pick $(x_\nu',x_\nu'')\in E_\nu' \times E_\nu''$ for each $\nu$. 
We can replace the third constraints by
\begin{align}\label{eq:whitney-lp3}
|\pl^\al (P^{x_\nu'} - P^{x_\nu''})(x_\nu')| &\le M|x_\nu'-x_\nu''|^{m-|\al|} &\forall x_\nu',x_\nu'', 1\le \nu',\nu''\le \nu_{\max} \text{ distinct}, |\al|\le m.
\end{align}
This is similar to the proof that when you estimate the Lipschitz constant, you can just estimate it over the representatives.
We have 
\begin{align}
\diam E_\nu' + \diam E_\nu'' &\le ad(E_\nu' ,E_\nu'') ,
\end{align}
and $P^x$ for each $x\in E$. Given~\eqref{eq:whitney-lp3}, we will prove that
\begin{align}
|\pl^\al (P^{x'}-P^{x''})(x')| &\le 1.01|x'-x''|^{m-|\al|}
\end{align}
for any $x',x''\in E$, $|\al|< m$.  %already assuming highest order derviatives bounded
Suppose not. Pick a counterexample $(x',x'',\al_0)$ with $|x'-x''|$ as small as possible.
\begin{align}
(x',x'')&\in E\times E\bs \text{Diag}\\
(x',x''), (x_\nu',x_\nu'') &\in E_\nu' \times E_\nu''
\end{align}
By minimality, $|x'-x_\nu'|,|x''-x_\nu''|\le a|x'-x''|$. 
\begin{align}
|\pl^\al(P^{x'} - P^{x_\nu'})(x_\nu')|
&\le |x'-x_\nu'|^{m-|\al|}\le a |x_\nu'-x_\nu''|^{m-|\al|}\\
|\pl^\al(P^{x''} - P^{x_\nu''})(x_\nu'')|
&\le |x''-x_\nu''|^{m-|\al|}\le \al |x_\nu'-x_\nu''|^{m-|\al|}\\
|\pl^\al(P^{x_\nu'} - P^{x_\nu''})(x_\nu')|
&\le |x_\nu'-x_\nu''|^{m-|\al|}
\end{align}
We can move the point at which we evaluate the polynomial.  We get 
\begin{align}
|\pl^\al(P^{x''} - P^{x_\nu''})(x_\nu')|
&\le Ca |x_\nu'-x_\nu''|^{m-|\al|}\\
|\pl^\al(P^{x'} - P^{x''})(x_\nu')|
&\le (L+Ca) |x_\nu'-x_\nu''|^{m-|\al|}
\end{align}
for $|\al|\le m$. (We actually need a stronger version of ``moving the basepoint''.)


%ineq for lin f'ls
%$|x_\nu'-x_\nu''|^{m-|\al|}M \pm \pl^\al (P^{x_\nu'} - P^{x_\nu''})(x_\nu')\ge 0$
The number of steps to solve a LP of size $n$ is $\poly(n)$, like $n^3$. But we will get it down to $n\log n$.

